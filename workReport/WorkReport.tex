\documentclass[pdf,titlepage,a4paper]{report}

\usepackage{graphicx}

%\usepackage{fancyhdr}
%\setlength{\headheight}{15.2pt}
%\pagestyle{fancy}
%\rhead{\includegraphics[scale=0.1]{Graphics/BASU_Logo_header.png}}
%\chead{فاز یک پروژه پیشرفته}

\usepackage[hidelinks]{hyperref}

\usepackage[fontsize=18pt]{fontsize}

\usepackage{xepersian}
\settextfont{B Zar}

%\newcommand{\s}{\fontdimen2\font=0.125em}

\begin{titlepage}
	\title{\includegraphics[scale=0.5]{Graphics/BASU_Logo_header.png} \\ \Huge{فاز یک پروژه پیشرفته}}
	\author{متین امیرپناه فر \\ شماره داشنجویی : 40212358003  \and \\ نیما مخملی \\ شماره داشنجویی : 40212358035}
	\date{\today}
\end{titlepage}

\begin{document}
	\maketitle
	\tableofcontents
	
	\begin{abstract}
	 
     \\در دورانی که قلمرو شاهی به استان های خود مختار تقسیم شده،سردارانی از گوشه ی سرزمین در سودای کشورگشایی،ارتش خود را برای فتح شهرها گسیل داشته اند.تا با مهارت و شجاعت بجنگند و با تدبیر و مکر به حریفان رکب بزنند تا قدرت و افتخار کسب کنند
     بازی رکب یک بازی تخته ای است که توسط 3 تا 6 بازیکن انجام میشود 
	 این بازی شامل دو نوع کارت بنفش و زرد است که هر کارت بنفش امتیاز خاص خود را دارد ولی کارت های زرد ویژگی خاصی ندارند
	 کارت های بنفش عبارتند از : مترسک , طبل زن , شاهدخت ,  شیرزن  ,  بهار  , زمستان , ریش سفید 
	 بازیکنان باید با کارت هایی که در دست دارند به جنگ بپردازند و سعی کنند در جنگ پیروز شوند و اگر سه استان مجاور یا پنج استان را فتح کنند پیروز کل بازی خواهند شد 
	 ما این بازی را به صورت شی گرا نوشته ایم که شرح کلاس ها و قابلیت های آنها را در ادامه برایتان میگوییم 
	\end{abstract}


	\part{کلاس ها}
		
	\newpage
	\section{بازی}
	این کلاس وظیفه اصلی و مدیریت کلی بازی را بر عهده دارد شرایط بازی را آماده و بازیکنان را مجهز و آماده جنگ میکند
	این کلاس با کلاس های بازیکن , زمین بازی , رابط کاربری , تمام کلاس های کارت ها و تمام کلاس های نشان رابطه دارد
	\paragraph{صفات}
	صفات این بازی عبارتند از آرایه ای از تمام کارت های بازی مانند آرایه هایی از کارت های بنفش با تعداد مختلف
	همچنین اشیاء مورد نیاز این کلاس از کلاس هایی مانند \lr{ui} , \lr{gameboard} , \lr{battlemarker} , \lr{player}

	\subparagraph{رابط کاربری}
	این کلاس وظیفه برقراری ارتباط سیستم و کاربران را دارد دستورات را از ورودی میگیرد و گزارشات را در خروجی برای رویت کردن کاربران چاپ میکند\\
    ین کلاس صفتی که به شکل خصوصی باشد ندارد 
	\paragraph{متدها}
	متد ها یا رفتار های این کلاس عبارتند از 
	\lr{clearTerminal} , \lr{pause} , \lr{spliter} , \lr{showPlayerPlayedCards} , \lr{getCommand} , \lr{get\_battleground} , \lr{showPlayerStates} , 
	\lr{showPlayerCards} , \lr{get\_card\_name} , \lr{get\_players\_number} , \lr{get\_player\_name} , \lr{get\_player\_old} , \lr{get\_player\_color} , 
	\lr{declare\_warWinner} , \lr{declare\_gameWinner}
	\subparagraph{شروع بازی}
	
	
	\newpage
	\section{رابط گرافیکی}
	 این کلاس در فاز های بعدی پروژه اضافه خواهد شد. \\
	\paragraph{صفات}
	\paragraph{متدها}


	\newpage
	\section{زمین بازی}
    این کلاس ارتباط بین استان هارا معلوم کرده و در واقع همان صفحه کاغذی موجود در بازی است \\
    این کلاس برای ما استان های بازی را نگهداری کرده و ارتباط بین آنها را ایجاد میکند و مشخص میکند که آیا دو استان با هم مجاورند یا خیر 

	\paragraph{صفات}
	صفات خصوصی این کلاس استان های بازی هستند که شامل 14 استان میشوند 
	این استان ها با استفاده از \lr{unordered\_map} و کلاس استان گرفته شده اند
	و صفت بعدی ارتباط بین این استان هاست که برای این کار از کلاس \lr{map} استفاده کرده ایم
	\paragraph{متدها}
	
	متد های این کلاس عبارتند از :

	\begin{latin}
		\begin{itemize}
			\item checkAdjacency برای چک کردن 
			\item getState ارسال استان های بازی 
			\item get\_active\_states\_name ارسال نام استان فعال و منتخب
		\end{itemize}
	\end{latin}
	
	\newpage
	\section{کارت ها}
	این کلاس مسئول نگهداری و سازماندهی کارت هاست این کلاس دو زیر کلاس دارد که بیانگر دو نوع کارت بنفش و زرد ایست که در بازی داریم

	\paragraph{صفات}
	صفات کلاس کارت ها  عبارتند از 
	1 - امتیاز که این کلاس چون دو کلاس از آن ارث بری میکنند صفت امتیاز را به صورت \lr{protected}  نوشته ایم
	2 - و ویژگی اولویت که به صورت خصوصی است

	\paragraph{متدها}
	توابع که برای این کلاس تعریف کرده ایم :
	\begin{latin}
		\begin{itemize}
			\item getType ارسال نوع کارت مدنظر
			\item applyFeature ویژگی کارت های ویژه یا بنفش که برای هر کارت به صورت جداگانه تعریف شده است
			\item is\_season چک کردن اینکه آیا کارت فصل است یا خیر؟
			\item getPriority ارسال اولویت کارت برای امور بازی 
			\item setPoint  قرار دادن امتیاز
			\item getPoint ارسال امتیاز مربوط به کارت
		\end{itemize}
	\end{latin}
	
	
	\subsection{کارت های بنفش(ویژه)}
	این کلاس برای کارت های ویژه این بازی طراحی شده است که تمام کلاس های شخصیت های کلاس بنفش از این کلاس ارث بری میکنند
	کلاس هایی همچون \lr{bishop} , \lr{drummer} , \lr{heroine} , \lr{scarecrow} , \lr{spring} , \lr{spy} , \lr{turncoat} , \lr{winter}  
	\paragraph{صفات}
	این کلاس فقط یک صفت دارد که آن هم برای \lr{type}  کارتهاست
	\subparagraph{}
	
	\paragraph{متدها}
	همچنین این کلاس فقط یک تابع برای ارسال \lr{type} دارد که اسم آن \lr{getType} است
	\subsubsection{مترسک}
	کلاس شخصیت مترسک : این کارت وقتی در بازی به کار برده میشود بازیکن میتواند یک کارت زرد را به بازی برگرداند 
	این کارت روی کارت های بنفش تاثیری ندارد
	\paragraph{صفات}
	 تنها  صفت این کلاس صفت \lr{help}  است که موقعی به کار برده میشود که کاربر بخواهد توضیحات مربوط به کارت را ببیند و با کارت آشنا شود
	 
	 \paragraph{متدها}
	 توابع این کلاس عبارتند از  
	 \lr{gethelp} که برای ارسال متن توضیحات است 
	 و \lr{applyFeature} که مربوط به ویژگی کارت است
	\subsubsection{طبل زن}
	 کلاس شخصیت طبل زن : این کارت وقتی توسط یک بازیکن رو میشود ارزش تمام کارت های زردی که آن بازیکن در یک درست آورده دو برابر میشود. البته فقط یکبار امکان پذیر است.

    \paragraph{صفات}
	تنها  صفت این کلاس صفت \lr{help}  است که موقعی به کار برده میشود که کاربر بخواهد توضیحات مربوط به کارت را ببیند و با کارت آشنا شود
	\paragraph{متدها}
	توابع این کلاس عبارتند از  
	\lr{gethelp} که برای ارسال متن توضیحات است 
	و \lr{applyFeature} که مربوط به ویژگی کارت است
	\subsubsection{شاهدخت}
	 کلاس شخصیت شاهدخت : کارت شاهدخت یک کارت امتیازیه که هیچ کارت دیگری روی آن اثر ندارد.
	\paragraph{صفات}
	این کلاس فقط یک صفت دارد به نام \lr{help}  که برای نمایش توضیحات این کارت است
	\paragraph{متدها}
	 دو متد این کلاس هم \lr{gethelp} , \lr{applyFeature} است.
	
	\subsection{فصل ها}
    این کلاس برای فصل های بهار و زمستان که دو کارت ویژه هستند طراحی شده است که همان طور که معلوم است دو کلاس بهار و زمستان از آن ارث بری میکنند

	\paragraph{صفات}
	این کلاس هیچ عضو داده ای خصوصی ندارد.
	\paragraph{متدها}
	تنها تابع عضو داده ای این کلاس تابع \lr{is\_season} است.
	
	\subsubsection{بهار}
	 کلاس شخصیت بهار : تا زمانی که بهار باشد به مجموع کارت های رو شده بازیکن یا بازیکنانی که بالا ترین کارت عددی را رو کرده اند یا  در آینده بازی میکنن را3 واحد اضافه میکند.

	\paragraph{صفات}
	این کلاس هم مانند تمام کلاس های کارت بنفش فقط یک صفت \lr{help} دارد
	\paragraph{متدها}
	\lr{gethelp} , \lr{applyFeature} .
	\subsubsection{زمستان}
	کارت شخصیت زمستان : تا زمانی که زمستان باشد عدد تمام کارت های زرد یا سرباز همه بازیکنان 1 میشود اگر بهار باشد و زمستان بیاید , زمستان میشود
	\paragraph{صفات}
	\lr{help}
	\paragraph{متدها}
	\lr{gethelp} , \lr{applyFeature}
	
	\subsection{کارت های زرد(سرباز)}
	کارت های زرد یا سرباز موقعی که رو میشوند قدرت ارتش بازیکن را به اندازه عدد روی کارت افزایش میدهد.

	\paragraph{صفات}
	هیچ صفتی ندارد 
	\paragraph{متدها}
	این کلاس هم دو تابع \lr{gethelp} , \lr{applyFeature} را دارا میباشد
	\newpage
	
	\section{نشان}
	کلاس نشان یک کلاس پدر است که سه کلاس فرزند دارد این کلاس را از آن جهت طراحی کرده ایم که نشان های بازیکن و جنگ و صلح را پیاده سازی کنیم
	\paragraph{صفات}
     صفات این کلاس :
	 1- یک شی از کلاس \lr{State}  برای اعمال نشان جنگ
	 2- یک شی از کلاس \lr{Color} برای پیاده سازی نشان بازیکنان با رنگ های متفاوت
	\paragraph{متدها}
	 توابع عضو این کلاس :
	 \begin{latin}
		\begin{itemize}
			\item setState قرار دادن استان مدنظر
	 		\item getState ارسال استان مدنظر 
	 		\item is\_set برای چک کردن که آیا استان منتخب است؟
		\end{itemize}
	 \end{latin}
	 
	\subsection{نشان بازیکن}
	این کلاس را برای نشان بازیکن طراحی کرده ایم تا بازیکنان به هنگام آغاز بازی یک نشان برای خود انتخاب کنند و وقتی یک استان را فتح کردند آن نشان روی استان مورد نظر قرار گیرد
	\paragraph{صفات}
	ندارد
	\paragraph{متدها}
	ندارد
	
	\subsection{ایالت}
	کلاس ایالت یا استان به این منظور طراحی شده که فرایندی که لازم است مستقیم با استان ها انجام شود راحت تر صورت بگیرند 

	\paragraph{صفات}
	 برای این کلاس دو صفت در نظر گرفته شده است
	 \begin{latin}
	 	\begin{itemize}
	 		\item name نام استان 
	 		\item set\_marker قرار دادن نشان فاتح استان
	 	\end{itemize}
	 \end{latin}
	\paragraph{متدها}
	
	\section{بازیکن}
	این کلاس به منظور مدیریت بازیکنان و تمام رفتار ها و ویژگی های آنها طراحی شده است 
	 این کلاس با کلاس های کارت استان و نشان بازیکن و رابط کاربری رابطه دارد
	\paragraph{صفات}
	ویژگی های خصوصی کلاس بازیکنان شامل نام , سن , آی دی , وکتوری از کارت های در اختیار ,وکتوری از کارت های بازی شده , استان های فتح شده و آرایه ای از نشان بازیکنان.

	\paragraph{متدها}
	توابع عضو این کلاس شامل توابع زیر میشوند :
	
	\begin{latin}
		\begin{itemize}
			\item getName برای ارسال نام بازیکنان
			\item getID برای ارسال شماره بازیکنان
			\item getAge  برای ارسال سن بازیکنان
			\item getPlayedCards برای ارسال کارت های بازی شده بازیکن
			\item getCards برای ارسال کارت های در دست بازیکنان
			\item setCards قرار دادن کارت ها
			\item setState قرار دادن استان مدنظر
			\item get\_states\_name ارسال نام استان
			\item drawn\_card کشیدن کارت برای بازی 
			\item push\_to\_cards افزودن کارت به کارت های در دست بازیکن
			\item push\_to\_playedCards افزودن به کارت های مصرف شده بازیکن
			\item drawn\_playedCard کشیدن کارت های بازی شده
		\end{itemize}
	\end{latin}
	
	\newpage
	
	
	
	\part{چالش ها}
	\\چالش هایی که در برنامه نویسی این بازی با آنها مواجه شدیم عبارتند از 
	1- برقراری ارتباط بین کلاس ها \\
	2- پیاده سازی مجاورت بودن استان های بازی که از کلاس \lr{map}  استفاده کردیم \\
	3- پیاده سازی ماهیت استان ها با استفاده از \lr{unordered\_map} \\
	4- کامپایل کردن و اجرای برنامه توسط \lr{CMAKE} "(اجرای برنامه ای با چندین فایل)"\\
    5- ایجاد بستری برای اجرای برنامه توسط رابط گرافیکی \lr{QT}\\

	\newpage
		
	\part{پیوندها}
	\section{گیت هاب}
	\href{https://github.com/Matin0789/Condottiere-.git}{لینک}

	\newpage
	
	
	
	\part{منابع}
	کتاب برنامه نویسی به زبان سی پلاس پلاس اثر دایتل و دایتل ویرایش دهم
	\href{https://www.geeksforgeeks.org/c-plus-plus/}{لینک}
\end{document}
